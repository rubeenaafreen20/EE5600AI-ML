\documentclass[journal,12pt,twocolumn]{IEEEtran}

\usepackage{setspace}
\usepackage{gensymb}

\singlespacing


\usepackage[cmex10]{amsmath}

\usepackage{amsthm}

\usepackage{mathrsfs}
\usepackage{txfonts}
\usepackage{stfloats}
\usepackage{bm}
\usepackage{cite}
\usepackage{cases}
\usepackage{subfig}

\usepackage{longtable}
\usepackage{multirow}

\usepackage{enumitem}
\usepackage{mathtools}
\usepackage{steinmetz}
\usepackage{tikz}
\usepackage{circuitikz}
\usepackage{verbatim}
\usepackage{tfrupee}
\usepackage[breaklinks=true]{hyperref}

\usepackage{tkz-euclide}

\usetikzlibrary{calc,math}
\usepackage{listings}
    \usepackage{color}                                            %%
    \usepackage{array}                                            %%
    \usepackage{longtable}                                        %%
    \usepackage{calc}                                             %%
    \usepackage{multirow}                                         %%
    \usepackage{hhline}                                           %%
    \usepackage{ifthen}                                           %%
    \usepackage{lscape}     
\usepackage{multicol}
\usepackage{chngcntr}

\DeclareMathOperator*{\Res}{Res}

\renewcommand\thesection{\arabic{section}}
\renewcommand\thesubsection{\thesection.\arabic{subsection}}
\renewcommand\thesubsubsection{\thesubsection.\arabic{subsubsection}}

\renewcommand\thesectiondis{\arabic{section}}
\renewcommand\thesubsectiondis{\thesectiondis.\arabic{subsection}}
\renewcommand\thesubsubsectiondis{\thesubsectiondis.\arabic{subsubsection}}


\hyphenation{op-tical net-works semi-conduc-tor}
\def\inputGnumericTable{}                                 %%

\lstset{
%language=C,
frame=single, 
breaklines=true,
columns=fullflexible
}
\begin{document}
\newtheorem{theorem}{Theorem}[section]
\newtheorem{problem}{Problem}
\newtheorem{proposition}{Proposition}[section]
\newtheorem{lemma}{Lemma}[section]
\newtheorem{corollary}[theorem]{Corollary}
\newtheorem{example}{Example}[section]
\newtheorem{definition}[problem]{Definition}

\newcommand{\BEQA}{\begin{eqnarray}}
\newcommand{\EEQA}{\end{eqnarray}}
\newcommand{\define}{\stackrel{\triangle}{=}}
\bibliographystyle{IEEEtran}
\providecommand{\mbf}{\mathbf}
\providecommand{\pr}[1]{\ensuremath{\Pr\left(#1\right)}}
\providecommand{\qfunc}[1]{\ensuremath{Q\left(#1\right)}}
\providecommand{\sbrak}[1]{\ensuremath{{}\left[#1\right]}}
\providecommand{\lsbrak}[1]{\ensuremath{{}\left[#1\right.}}
\providecommand{\rsbrak}[1]{\ensuremath{{}\left.#1\right]}}
\providecommand{\brak}[1]{\ensuremath{\left(#1\right)}}
\providecommand{\lbrak}[1]{\ensuremath{\left(#1\right.}}
\providecommand{\rbrak}[1]{\ensuremath{\left.#1\right)}}
\providecommand{\cbrak}[1]{\ensuremath{\left\{#1\right\}}}
\providecommand{\lcbrak}[1]{\ensuremath{\left\{#1\right.}}
\providecommand{\rcbrak}[1]{\ensuremath{\left.#1\right\}}}
\theoremstyle{remark}
\newtheorem{rem}{Remark}
\newcommand{\sgn}{\mathop{\mathrm{sgn}}}
\providecommand{\abs}[1]{\left\vert#1\right\vert}
\providecommand{\res}[1]{\Res\displaylimits_{#1}} 
\providecommand{\norm}[1]{\left\lVert#1\right\rVert}
%\providecommand{\norm}[1]{\lVert#1\rVert}
\providecommand{\mtx}[1]{\mathbf{#1}}
\providecommand{\mean}[1]{E\left[ #1 \right]}
\providecommand{\fourier}{\overset{\mathcal{F}}{ \rightleftharpoons}}
%\providecommand{\hilbert}{\overset{\mathcal{H}}{ \rightleftharpoons}}
\providecommand{\system}{\overset{\mathcal{H}}{ \longleftrightarrow}}
	%\newcommand{\solution}[2]{\textbf{Solution:}{#1}}
\newcommand{\solution}{\noindent \textbf{Solution: }}
\newcommand{\cosec}{\,\text{cosec}\,}
\providecommand{\dec}[2]{\ensuremath{\overset{#1}{\underset{#2}{\gtrless}}}}
\newcommand{\myvec}[1]{\ensuremath{\begin{pmatrix}#1\end{pmatrix}}}
\newcommand{\mydet}[1]{\ensuremath{\begin{vmatrix}#1\end{vmatrix}}}
\numberwithin{equation}{subsection}
\makeatletter
\@addtoreset{figure}{problem}
\makeatother
\let\StandardTheFigure\thefigure
\let\vec\mathbf
\renewcommand{\thefigure}{\theproblem}
\def\putbox#1#2#3{\makebox[0in][l]{\makebox[#1][l]{}\raisebox{\baselineskip}[0in][0in]{\raisebox{#2}[0in][0in]{#3}}}}
     \def\rightbox#1{\makebox[0in][r]{#1}}
     \def\centbox#1{\makebox[0in]{#1}}
     \def\topbox#1{\raisebox{-\baselineskip}[0in][0in]{#1}}
     \def\midbox#1{\raisebox{-0.5\baselineskip}[0in][0in]{#1}}
\vspace{3cm}
\title{Assignment 2}
\author{Rubeena Aafreen (EE20RESCH11012)}
\maketitle
\newpage
\bigskip
\renewcommand{\thefigure}{\theenumi}
\renewcommand{\thetable}{\theenumi}
Download all python codes from 
\begin{lstlisting}
https://github.com/rubeenaafreen20/EE5600AI-ML/tree/master/Assignment2/Code
\end{lstlisting}
%
and latex codes from 
%
\begin{lstlisting}
https://github.com/rubeenaafreen20/EE5600AI-ML/tree/master/Assignment2
\end{lstlisting}
%
\section{Problem}
On one page of a telephone directory, there were 200 telephone numbers. The frequency distribution of their unit place digit (for example, in the number 25828573, the unit place digit is 3) is given in Table below:
\begin{table}[!ht]
\renewcommand\thetable{1}
\centering
% Table generated by Excel2LaTeX from sheet 'given_freq_dist'

    \begin{tabular}{|r|r|r|}
    \toprule
    \hline
    \multicolumn{1}{|c|}{\textbf{Digit}} & \multicolumn{1}{c|}{\textbf{Frequency}} & \multicolumn{1}{c|}{\textbf{Probability}} \\
    \midrule
    \hline
    0     & 22    & 0.11 \\
    1     & 26    & 0.13 \\
    2     & 22    & 0.11 \\
    3     & 22    & 0.11 \\
    4     & 20    & 0.1 \\
    5     & 10    & 0.05 \\
    6     & 14    & 0.07 \\
    7     & 28    & 0.14 \\
    8     & 16    & 0.08 \\
    9     & 20    & 0.1 \\
    \hline
    \end{tabular}%

\caption{Given frequency distribution}
\label{tab:table1}
\end{table}
Without looking at the page, the pencil is placed on one of these numbers, i.e., the number is chosen at random. What is the probability that the digit in its unit place is 6?
\section{Explanation}
probability is defined as
\begin{align}
    P=\frac{\text{number of outcomes}}{\text{Sample space}}
\end{align}
\section{Solution}
Let $X\in\{i\}\math{_{i=1}^{i=6}}$ and ${f_i}$ be the corresponding frequency. Then,
\begin{align}
    P_r (X=i)= \dfrac{{f}_{i}}{200}
\end{align}
From table \ref{tab:table1},
\begin{align}
    P_r (X=6)= \dfrac{14}{200}\\
    =0.07
\end{align}
\section{Output}
The outputs of Python program are attached below:
\begin{table}[!ht]
\renewcommand\thetable{2}
\centering
% Table generated by Excel2LaTeX from sheet 'sheet_random'

    \begin{tabular}{|r|r|r|}
    \toprule
    \hline
    \multicolumn{1}{|c|}{\textbf{Digit}} & \multicolumn{1}{c|}{\textbf{Frequency}} & \multicolumn{1}{c|}{\textbf{Probability}} \\
    \hline
    \midrule
    0     & 21    & 0.105 \\
    1     & 13    & 0.065 \\
    2     & 20    & 0.1 \\
    3     & 21    & 0.105 \\
    4     & 20    & 0.1 \\
    5     & 25    & 0.125 \\
    6     & 15    & 0.075 \\
    7     & 24    & 0.12 \\
    8     & 20    & 0.1 \\
    9     & 21    & 0.105 \\
    \hline
    \end{tabular}%


\caption{For 200 randomly generated numbers}
\label{tab:table2}
\end{table}

\begin{table}[!ht]
\renewcommand\thetable{3}
\centering
% Table generated by Excel2LaTeX from sheet 'sheet_random_2'

    \begin{tabular}{|r|r|r|}
    \toprule
    \hline
    \multicolumn{1}{|c|}{\textbf{Digit}} & \multicolumn{1}{c|}{\textbf{Frequency}} & \multicolumn{1}{c|}{\textbf{Probability}} \\
    \midrule
    \hline
    0     & 1007  & 0.1007 \\
    1     & 988   & 0.0988 \\
    2     & 997   & 0.0997 \\
    3     & 1010  & 0.101 \\
    4     & 1005  & 0.1005 \\
    5     & 1018  & 0.1018 \\
    6     & 1000  & 0.1 \\
    7     & 984   & 0.0984 \\
    8     & 1019  & 0.1019 \\
    9     & 972   & 0.0972 \\
    \hline
    \end{tabular}%

\caption{For 10000 randomly generated numbers}
\label{tab:table3}
\end{table}
\section{Explanation}
The \textbf{Law Of Large Numbers} is a fundamental concept for probability and statistics. It states that  that as the number of trials increase, the experimental probability will get closer and closer to the theoretical probability.
\newline From the output tables \ref{tab:table2} and \ref{tab:table3}, we can deduce that as the number of trials increase,  the ratio of the number of successful occurrences to the number of trials will tend to approach the theoretical probability of the outcome for an individual trial. 
Since all the digits are equiprobable, ideally each probability should be 1/10=0.1
\newline In table \ref{tab:table3}, when number of trials are 10,000, probability of each digit is approximately 0.1 with very little deviation. eg. 0.1005.  
\newline
With 200 samples, Tables \ref{tab:table2} and \ref{tab:table3} are slightly different because the number of simulations is not sufficient for convergence in the probabilities.
\end{document}