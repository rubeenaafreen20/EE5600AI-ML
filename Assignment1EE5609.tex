\documentclass[journal,12pt,twocolumn]{IEEEtran}

\usepackage{setspace}
\usepackage{gensymb}

\singlespacing


\usepackage[cmex10]{amsmath}

\usepackage{amsthm}

\usepackage{mathrsfs}
\usepackage{txfonts}
\usepackage{stfloats}
\usepackage{bm}
\usepackage{cite}
\usepackage{cases}
\usepackage{subfig}

\usepackage{longtable}
\usepackage{multirow}

\usepackage{enumitem}
\usepackage{mathtools}
\usepackage{steinmetz}
\usepackage{tikz}
\usepackage{circuitikz}
\usepackage{verbatim}
\usepackage{tfrupee}
\usepackage[breaklinks=true]{hyperref}

\usepackage{tkz-euclide}

\usetikzlibrary{calc,math}
\usepackage{listings}
    \usepackage{color}                                            %%
    \usepackage{array}                                            %%
    \usepackage{longtable}                                        %%
    \usepackage{calc}                                             %%
    \usepackage{multirow}                                         %%
    \usepackage{hhline}                                           %%
    \usepackage{ifthen}                                           %%
    \usepackage{lscape}     
\usepackage{multicol}
\usepackage{chngcntr}

\DeclareMathOperator*{\Res}{Res}

\renewcommand\thesection{\arabic{section}}
\renewcommand\thesubsection{\thesection.\arabic{subsection}}
\renewcommand\thesubsubsection{\thesubsection.\arabic{subsubsection}}

\renewcommand\thesectiondis{\arabic{section}}
\renewcommand\thesubsectiondis{\thesectiondis.\arabic{subsection}}
\renewcommand\thesubsubsectiondis{\thesubsectiondis.\arabic{subsubsection}}


\hyphenation{op-tical net-works semi-conduc-tor}
\def\inputGnumericTable{}                                 %%

\lstset{
%language=C,
frame=single, 
breaklines=true,
columns=fullflexible
}
\begin{document}


\newtheorem{theorem}{Theorem}[section]
\newtheorem{problem}{Problem}
\newtheorem{proposition}{Proposition}[section]
\newtheorem{lemma}{Lemma}[section]
\newtheorem{corollary}[theorem]{Corollary}
\newtheorem{example}{Example}[section]
\newtheorem{definition}[problem]{Definition}

\newcommand{\BEQA}{\begin{eqnarray}}
\newcommand{\EEQA}{\end{eqnarray}}
\newcommand{\define}{\stackrel{\triangle}{=}}
\bibliographystyle{IEEEtran}
\providecommand{\mbf}{\mathbf}
\providecommand{\pr}[1]{\ensuremath{\Pr\left(#1\right)}}
\providecommand{\qfunc}[1]{\ensuremath{Q\left(#1\right)}}
\providecommand{\sbrak}[1]{\ensuremath{{}\left[#1\right]}}
\providecommand{\lsbrak}[1]{\ensuremath{{}\left[#1\right.}}
\providecommand{\rsbrak}[1]{\ensuremath{{}\left.#1\right]}}
\providecommand{\brak}[1]{\ensuremath{\left(#1\right)}}
\providecommand{\lbrak}[1]{\ensuremath{\left(#1\right.}}
\providecommand{\rbrak}[1]{\ensuremath{\left.#1\right)}}
\providecommand{\cbrak}[1]{\ensuremath{\left\{#1\right\}}}
\providecommand{\lcbrak}[1]{\ensuremath{\left\{#1\right.}}
\providecommand{\rcbrak}[1]{\ensuremath{\left.#1\right\}}}
\theoremstyle{remark}
\newtheorem{rem}{Remark}
\newcommand{\sgn}{\mathop{\mathrm{sgn}}}
\providecommand{\abs}[1]{\left\vert#1\right\vert}
\providecommand{\res}[1]{\Res\displaylimits_{#1}} 
\providecommand{\norm}[1]{\left\lVert#1\right\rVert}
%\providecommand{\norm}[1]{\lVert#1\rVert}
\providecommand{\mtx}[1]{\mathbf{#1}}
\providecommand{\mean}[1]{E\left[ #1 \right]}
\providecommand{\fourier}{\overset{\mathcal{F}}{ \rightleftharpoons}}
%\providecommand{\hilbert}{\overset{\mathcal{H}}{ \rightleftharpoons}}
\providecommand{\system}{\overset{\mathcal{H}}{ \longleftrightarrow}}
	%\newcommand{\solution}[2]{\textbf{Solution:}{#1}}
\newcommand{\solution}{\noindent \textbf{Solution: }}
\newcommand{\cosec}{\,\text{cosec}\,}
\providecommand{\dec}[2]{\ensuremath{\overset{#1}{\underset{#2}{\gtrless}}}}
\newcommand{\myvec}[1]{\ensuremath{\begin{pmatrix}#1\end{pmatrix}}}
\newcommand{\mydet}[1]{\ensuremath{\begin{vmatrix}#1\end{vmatrix}}}
\numberwithin{equation}{subsection}
\makeatletter
\@addtoreset{figure}{problem}
\makeatother
\let\StandardTheFigure\thefigure
\let\vec\mathbf
\renewcommand{\thefigure}{\theproblem}
\def\putbox#1#2#3{\makebox[0in][l]{\makebox[#1][l]{}\raisebox{\baselineskip}[0in][0in]{\raisebox{#2}[0in][0in]{#3}}}}
     \def\rightbox#1{\makebox[0in][r]{#1}}
     \def\centbox#1{\makebox[0in]{#1}}
     \def\topbox#1{\raisebox{-\baselineskip}[0in][0in]{#1}}
     \def\midbox#1{\raisebox{-0.5\baselineskip}[0in][0in]{#1}}
\vspace{3cm}
\title{Assignment 1}
\author{Rubeena Aafreen (EE20RESCH11012)}
\maketitle
\newpage
\bigskip
\renewcommand{\thefigure}{\theenumi}
\renewcommand{\thetable}{\theenumi}
Download all python codes from 
\begin{lstlisting}
https://github.com/rubeenaafreen20/EE5609/tree/master/Codes
\end{lstlisting}
%
and latex codes from 
%
\begin{lstlisting}
https://github.com/rubeenaafreen20/EE5609
\end{lstlisting}
%
\section{Problem}
A ray of light passing through the point \myvec{
1\\
2
}
reflects on the x-axis at point \textbf{A} and the reflected ray passes through the point \myvec{
5\\
3
}. Find
the coordinates of \textbf{A}.
\section{Explanation}
Let point \textbf{P} be \myvec{1\\2} and point \textbf{Q} be \myvec{5\\3}
Since, point \textbf{A} is on x-axis, ts y-coordinate is zero.
Assume \begin{align}
    A=\begin{pmatrix} k \\ 0 \end{pmatrix}
\end{align} \\
\begin{figure}[h]
\centering
\includegraphics[width=\columnwidth]{ques.jpeg}
\caption{Incident and reflected ray vectors}
\label{fig:}
\end{figure}
\\
Incident vector = \textbf{d} = P-A
\begin{align}
    \textbf{d}=\myvec{1-k \\ 2}
\end{align}
\\
Reflected vector = \textbf{r} = Q-A
\begin{align}
    \textbf{r}=\myvec{5-k \\ 3}
\end{align}
\\
Normal vector
\begin{align}
    \textbf{n}=\myvec{1\\0}
\end{align}
\\
From Fig. 0, \\
Projection of \Vec{d} in the direction of \Vec{n} is given by\\
\begin{align}
\Vec{d_{1}}=(\Vec{d}^{T}{\hat{\Vec{n}}})\hat{\Vec{n}}
\end{align}
Projection of \Vec{d} in the orthogonal direction is given by\\
\begin{align}
\Vec{d_{2}}=\vec{d}-(\Vec{d}^{T}{\hat{\Vec{n}}})\hat{\Vec{n}}
\end{align}

Projection of \textbf{b}= -(Projection of \textbf{d}) \\
\begin{align}
\Vec{b}=-(\Vec{d}^{T}{\hat{\Vec{n}}})\hat{\Vec{n}}+(\vec{d}-(\Vec{d}^{T}{\hat{\Vec{n}}})\hat{\Vec{n}})
\\
\implies \vec{b}=\vec{d}-2(\vec{d}^{T}\hat{\vec{n}})\hat{\vec{n}}\\
\implies \vec{b}=\vec{d}-2\dfrac{\vec{d}^{T}{\vec{n}}}{\norm{\vec{n}}^2}\vec{n}\\
\norm{\vec{n}^2}=1
\end{align}

Hence,equation for reflected vector can be written as:
\begin{align}
    \vec{b}=\vec{d}-2(\vec{d}^{T}\vec{n})\vec{n}
\end{align}

\section{Solution}
\\
Solving the equation (2.0.11):\\
\begin{align}
    \textbf{b}=\myvec{1-k\\2}-2\left(\myvec{k-1  & 2}\myvec{1\\0}\right).\myvec{1\\0}
\end{align}
\\ 
\begin{align}
    \implies \textbf{b}=\myvec{k-1 \\ 2}
\end{align}
\\
From equations (2.0.3) and (3.0.2), we get
\begin{align}
    k=\dfrac{13}{5}=2.6
\end{align}
\section{Verification}
\\
Putting k=2.6 in equations (2.0.3) and (3.0.2), the value of calculated reflected vector \textbf{d} and given reflected vector \textbf{d} are,
\begin{align}
\textbf{r}=\myvec{5-2.6 \\ 3}\\
\implies \textbf{r}=\myvec{2.4\\3}
\end{align}\\
and\\
\begin{align}
\textbf{b}=\myvec{2.6-1 \\ 2}\\
\implies \textbf{b}=\myvec{1.6\\2}
\end{align}\\
Value of k is correct if unit vectors of both \textbf{r} and \textbf{b} are same.
\\
\begin{align}
    b=\frac{\bm b}{\norm{\bm b}}= \frac{2 \myvec{0.8\\1}}{\sqrt{(1.6)^2+(2)^2}}
    \end{align}\\
    \begin{align}
    \implies b=0.78 \myvec{0.8\\1}
    \end{align}
and \\
\begin{align}
    r=\frac{\bm r}{\norm{\bm r}}= \frac{3 \myvec{0.8\\1}}{\sqrt{(2.4)^2+(3)^2}}
    \end{align}\\
    \begin{align}
    \implies b=0.78 \myvec{0.8\\1}
    \end{align}
    \\
    From equations (4.0.6) and (4.0.8), we observe that the solution is verified.\\
\\
\end{document}